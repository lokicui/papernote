% This is LLNCS.DOC the documentation file of
% the LaTeX2e class from Springer-Verlag
% for Lecture Notes in Computer Science, version 2.4
\documentclass{llncs}
\usepackage{llncsdoc}
%
\begin{document}
\markboth{\LaTeXe{} Class for Lecture Notes in ComputerScience}
{\LaTeXe{} Class for Lecture Notes in Computer Science}
\thispagestyle{empty}
\vspace{2pt}
\vfill

%
\title {Leveraging  Aging theory in topic-focused multi-document timeline summarization} 
\maketitle
%
\section{Abstract}

Topic-focused multi-document summarization plays an important role in helping readers to get the main information from any topic. Many approaches are proposed to generate the timeline summarization, but ignore the life circle of each topic. In this paper, aging theory is leveraged to compute the sentence feature in the first step, and then train the classification model with the SMOTEBoost technology. %We evaluate our approach on two corpora, one of which is a public data set, the other one is our manual annotation data set. Experiment results show that our method can improve the timeline summarization significantly.
Experiment results show that our method can improve the timeline summarization significantly.

%
\section{Introduction}
%


%Everyday thousands of news stories reporting different events are published on the Internet. These reports are disordered and people have to read most of them to know what is happening which is a time-consuming job undoubtedly. How can we get useful information about an event efficiently? Automatic summarization has been such a method solving this kind of information overloading since Luhn \cite{1958-Luhn-p159-165} proposed it in 1958. And numerous pages have been published in this field, ranging from single document to multiple documents, from extraction to abstraction, from traditional document to web document, email, blog and other types of genre. However, these research work focus on the central idea of document or document set ignoring the temporal characteristics of events. As a result, people cannot catch the changes of events over time efficiently. 

%Recent years, topic detection and tracking (TDT) which detects new events from the large scale news stream and tracks them as events going on draws researchers' attention. But it did not display events properly, and people still have to read all the relevant reports to get what they want to know about the event. However, we are still enlightened by its usage of tracking which makes us decide to generate a timeline summary consisting of a series of individual small summaries with sentences both important and diverse to help people understand the progress of an event more quickly.

Everyday thousands of news stories reporting different events are published on the internet. With the help of topic detection and tracking(TDT), people can know what events happened and this save a lot of time for readers. However, most news reports are not prepared for the progress of event, so that there are lots of duplication message among reports about the same event. Topic-focused multi-document summarization(TFMDS) aims to gain the main information from the topic-focused documents and the timeline summarization help to reorganize the order and sentences selection to get a better reading experience.

As we know, event goes through a life cycle of birth, growth, maturity and death, and this can be reflected from special terms utilized for descripting different events experience a similar life cycle. Aging theory \cite{chen2003life} is a model exploited in event detection task which tracks life cycles of events using energy function. The energy of an event increases when the event becomes popular, and it diminishes with time. In our opinion, it can also be used for summarization to help us find out the daily hot terms of events. Then people can obtain what new changes happen as events going on.

In the keywords-based summarization field, there are two main challenges for choosing summary sentences. One challenge lies in computing the importance of sentences, which is decided by terms occurring at the documents . But different authors use different words to express the same meaning and lots of synonyms. In order to find the core words in the news without the influence caused by the synonym and polysemy, we use latent semantic analysis (LSA) \cite{1990-Deerwester-p391-407} to handle the dataset. LSA is a robust unsupervised technique for deriving an implicit representation of text semantics based on observed co-occurrence of words to find semantic units of news.

The other challenge is how to handle the sentences. In each topic, only a few sentences will be labled as the summarization sentence, that is, the data set is imbalanced. This situation causes a problem that the training model will prefer the normal sentences. SMOTEBoost \cite{chawla2003smoteboost} combines SMOTE \cite{chawla2011smote} and boost technology in order to impove the precision of minority class when resamping. This has been proved effective for imbalanced data set. 

In this paper, we generate news event timeline summary by considering both temporal and semantic characteristics. We first extract the features from five aspects to represent each sentence. Then, classification model is built with SMOTEBoost. Last, we choose sentences from candidates to form the summary and display them with timeline, so that people can track the progress of event easily and quickly.

The remainder of this paper is organized as follows: Section 2 reviews some related works on summarization. We discuss our approach about how to leverage aging theory to gain sentence feature and train the classification model with SMOTEBoost technology in section 3. Our experiments and some discusses are described in section 4. Section 5 presents our conclusions and some future plans.


%
\section{Related Work}
%
\textbf{Topic-focused multi-document summarization(TFMDS)} aims at gaining main information from multiple texts about the same topic. There are two ways to achieve this goal, one is extract important sentences, the other one is build new sentence to express the key idea. In this paper, we focus on the former method.

One of the most popular extractive multi-document summarization methods is MEAD \cite{2004-Radev-p919-938}, which takes term frequency, sentence position, first-sentence overlap to present the feature of each sentence. Xiaojun Wan \cite{wan2007manifold} proposed an extractive approach based on manifold-ranking about the information richness and novelty. Wan \cite{2008-Wan-p299-306} used markov random walk model and cluster hits model to analysis the link relationships between sentences in order to gain the important one as the summary. Wong \cite{2008-Wong-p985-992} investigated co-training method by combining labeled and unlabeled data to train the model, which used four kinds of features can be categorized as surface, content, relevance and event features.

\textbf{Timeline summarization(TS)} gains enough attraction with the development of Topic Detection and Track(TDT). Lots of timeline summarization methods have been developed recently. ETS \cite{2011-Yan-p745-754} formulated the task as an optimization  problem via iterative substitution from a set of sentences with four requirements. Giang Binh Tran\cite{tran2013leveraging} investigated five different sentence features and leveraged SVMRank to optimize the summarization task. Wayne \cite{zhao2013timeline} took social attention involved to compute the importance. ETTS \cite{yan2011timeline} utilized trans-temporal characteristics to gain the summary. Giang \cite{binh2013structured} extracted the temporal information and surface features to train the regression model for predicting the summary sentences. Nedunchelian \cite{2008-Nedunchelian-p480-485} reused the MEAD and add the timestamp feature to implement their TS.
 
\textbf{Aging theory} has been proved effictive to track which stage of life cycle for news. Chen et al. \cite{2007-Chen-p1016-1025} \cite{2003-Chen-p47-59} applied this to model the news event's life cycle and utilized the concept of energy to track it. Because our aim is to gain the summary of multi-documents of news domain, we consider aging theory is worth using to extract the feature of sentence.

%
\section{Our Approach}
%

\subsection{Key Concepts}
\textbf{Topic-focused}: What we value most is an event grouped from several web news articles, such as the "the missing of the malaysia airlines plane" from BBC. These articles show us the cause, the progress and the results about the topic. Most of our summarrization technology's application scenario is working for TDT system. \textbf{Timeline Summaries:} Generally speaking, timeline is a kind of display forms for the summaries. But timeline summaries should show us the progress of this topic not just display according to the time sequence. Under this condition of the requirement, timeline summary of each day should describe the most important thing happened in that day.

We give the formal definition of topic-focused multi-document timeline summarization(TMTS) as follows:

\textbf{Input}: Given a set of documents $D=\{d_1, d_2, \dots, d_n\}$ which should cover progress of the topic in the time span $T=\{t_1, t_2, \dots, t_m\}$. We segment each document to sentences and group them by the date to form sentences $S=\{s_1, s_2, \dots, s_m\}$. 

\textbf{Output}: The TMTS should output the summaries along the date and each summary is the main idea of what occured in that day, i.e. $O=\{o_1, o_2, \dots, o_m\}$, where $o_i$ means the summary of sentences from all the documents of that day $s_i$. 

In order to respresent the most important thing happened in that day, the summary should consider the importance, the movelty, and contains the topic hot terms. Under the condition of this assumption, the features we use to represent the sentence as follows:

\subsection{Sentence Feature Selection}

In order to represent sentence, we extract five kinds of features as flows:

\textbf{Surface feature}: this contains features computed by basic statistics, such as the length of sentence, the counts of noun words and stop words, the position in this document and paragraph, and whether it contains person name or not. %$Weight_{surface} = \{Length_{sentence}, Count_{nourms}, Count_{stopwords}, Position_{indocument}, Position_{inparagrph}, Boolean_{containspersonname}\}$

\textbf{Importance feature}: this feature aims to respesent the importance of this sentence. The weight of sentence is computed through linear combination of term weights with latent semantic analysis. The function is 
\begin{equation}
Weight_{importance} = \sum_{w \in sentence}TF_w * LSA_w
\end{equation}

\textbf{Aging feature}: We use this feature to show the life cycle of the sentence. The frequency of a word will change as event going on, so we use the association between word  and time interval  to indicate its energy which is defined as follows:
\begin{equation}
  E_{w,t} =F(F^{-1}E_{w,t-1}+\alpha\cdot\chi^2_{w,t})
\end{equation}
where $E_{w,t}$ is the energy of word $w$  in time interval $t$ , and $E_{w,t-1}$  is the energy of word $w$ in time interval $t-1$ , $\alpha$ is the transfer factor, and $\chi^2_{w,t}$  is the contribution degree of word  at the time interval $t$, which can be computed as presented in \cite{2000-Swan-p49-56}. 

However, no words descripting a special event point will retain popular forever, they will decay over time. In order to represent the word's life span realistically, we cut down the energy of word by a decay factor β at the end of every time interval. And if the decayed energy value became negative, we change it to 0.

According to the description above, if the energies of some words increase greatly, we can draw a conclusion that there is a hot event spot. So we need to calculate the variance of word energy next. Here we use standard deviation:
\begin{equation}
Var_{w,t} = \sqrt{ \frac{1}{N} \sum_{t \in period}(E_{w,t}- \overline{E_{w}})^2}
\end{equation}
where $N$ is the number of time intervals during the given period, $E_{w,t}$  is the energy of word $w$  in time interval $t$ , $\overline{E_{w}}$  is the average energy during the period, and $Var_{w,t}$  is the variance of $w$ .
Then each word will be assigned a new weight besides the traditional TFIDF which can be defined as:
 \begin{equation}
Weight_{aging}=\sum_{w \in sentence_i}TF*IDF_{w} + \mu \cdot Var_{w}
\end{equation}
This kind of new weight can help us identify both central and hot information, so people can capture the main line and new changes of events simultaneously.

\textbf{Topic feature}: Each topic contains lots of sentences, that is, every sentence contribute some information to the whole set to express the topic. In this study, we consider use the link analysis to compute the latent semantic between sentences. First, topic terms and topic elements can be found through the word frequency analysis. Then, the similarity among sentences are computed with the cosine function to construct the event map. Last, pageRank algorithm is used to assigin the weight to each node in this map. We treat the pageRank value as the topic feature of the sentence. The function is :
\begin{equation}
Weight_{topic} = PageRank(sentence_i)
\end{equation}

\textbf{Novelty feature}: In order to avoid some sentences with the same meanings be selected as the summary, the novelty of sentence is important. The novelty value is compute by the distance with the summary of last time span. The larger the distance is, the more the novelty this sentence is. In our research, we use the Jaccard similarity to gain this. The function is :
\begin{equation}
Weight_{novelty} = 1 - Jaccard(sentence_i, summary_{ex})
\end{equation}


\subsection{Model Training}

With the help of labled data, we convert this summarization task to pairwise classification problem. The positive data is sentences labeled to summary, otherwise is negative. 

Because the count of summary sentence is much less than the normal sentences, the train data set is unbalanced. In order to reduce this reflect, SMOTEBoost method is used to train the classification model.

%
\section{Evaluation}
%

\subsection{Evaluation metric}

Here we use ROUGE toolkit \cite{2004-Lin-p74-81} , which is officially applied by Document Understanding Conference (DUC) for document summarization performance evaluation, to evaluate the experimental results and compare these algorithms with each other. The summarization quality is measured by counting the number of overlapping units, such as N-gram, word sequences, and word pairs between the auto-generated summary and the manual summary. Several automatic evaluation methods are implemented in ROUGE, such as ROUGE-N, ROUGE-L and ROUGE-W, each of which can generate three scores (recall (R), precision (P) %and F-measure%
). The function is:

\begin{equation}
R=\frac{\sum_{s \in manual } \sum_{N-gram \in s} Count_{match}(N-gram) } { \sum_{s \in manual} \sum_{N-gram \in s} Count(N-gram)  } \end{equation}
\begin{equation}
P=\frac{\sum_{s \in auto } \sum_{N-gram \in s} Count_{match}(N-gram) } { \sum_{s \in auto} \sum_{N-gram \in s} Count(N-gram)  }
\end{equation}

%\begin{equation}
%F-measure = \frac{2PR}{P+R}
%\end{equation}

Where $N$ stands for the length of the $N-gram$, $Countmatch(N-gram)$ is the maximum number of N-grams co-occurring in the auto-generated summary and the manual summary. As a result of all the training datas are labeled data, the precision and recall rate can be computed easily and these two metric are helpful get a quick understanding about the performance.
%Here we execute our evaluation in terms of ROUGE-1. And since our summary consists of a series of individual small summaries, we use the average ROUGE score as the final score.


\subsection{Methods to compare}

We start our experiment with some preprocessing like indexing, filtering out the stop words and segmenting news documents into sentences. Then we perform our method to the data set and generate a timeline for each event we choosing. We also implement some widely used multi-document summarization methods as the baselines.

$Centroid$ extracts sentences based on centroid value, positional value and first-sentence overlap.

$Cluster$ considers that there are different themes in an event, so it first clusters similar sentences together into different clusters and then selects one representative sentence from each main cluster.

$Allan$ is a similar timeline system from different aspects proposed by Allan et al., dividing sentences into on-event and off-event while ranking them with useful and novelty.

$Wong$ combined the supervised and semi-supervised learning and used co-training method to train the labeled data and unlabeled data. 

$L2RTS$ considered the summary task as optimistic task and  leveaged the SVMRank to gain the summary.

\subsection{Experiment on public data set}

%The public data set from Giang Binh Tran\cite{tran2013leveraging} is used in this research. This data set collects 17 timelines in 9 topics and obtains 4650 news articles. The data set provider takes the news editor's event summary as the topic summary, and some of labeled summaries are created by themselves. So there are many summary sentences are not extracted from the document, which brings extra problem about evaluate the results from TMTS.

%\begin{table}
%\caption{Performance on public data set}
%\begin{tabular}{|c|c|c|c|c|c|c|}
%\hline
%Method  & Precision(1) & Recall(1)& Precision(2) & Recall(2) \\
%\hline
%Centroid &				& 			 			&				& 			 				\\
%\hline
%Cluster	&				&						&				& 			 				\\
%\hline
%Allan		&				&						&				& 			 				\\
%\hline
%Wong		&				&						&				& 							\\
%\hline
%L2RTS	&				&						&				& 			 				\\
%\hline
%TMTS	&				&						&				& 							\\
%\hline
%\end{tabular}
%\end{table}

%\subsection{Experiment on manual labeled data}

%We also create a data set, which comes from the TAC2010. This data set contains 906 documents around 46 topics from the New York Times, the Associated Press, the Xinhua News Agency newswires, and the Washington Post News Service, however, it is not for summarization originally because of no summary labeled. In order to gain the relative accurate lable result, we distribute these documents to five people to label, then the highest count of sentence is the summary.
We labeled a dataset which comes from the TAC2010. This data set contains 906 documents around 46 topics from the New York Times, the Associated Press, the Xinhua News Agency newswires, and the Washington Post News Service, however, it is not for summarization originally because of no summary labeled. In order to gain the relative accurate lable result, we distribute these documents to five people to label, then the highest count of sentence is the summary.

\begin{table}
\caption{Performance on manual labeled data set}
\begin{tabular}{|c|c|c|c|c|c|c|}
\hline
Method  & Precision(1) & Recall(1)  & Precision(3) & Recall(3) \\
\hline
Centroid &	0.129			& 		0.076	 			&	0.228			& 0.129			 				\\
\hline
Cluster	&	0.057			&		0.045				&	0.185			& 	0.133		 				\\
\hline
Allan		&	0.143			&		0.083				&	0.232			& 0.184			 				\\
\hline
Wong		&	0.214			&		0.085				&	0.363			& 0.135							\\
\hline
L2RTS	&	0.221			&		0.095				&	0.394			& 0.157			 				\\
\hline
TMTS	&	0.233			&		0.090				&	0.391			& 0.159							\\
\hline
\end{tabular}
\end{table}

%From the two experiments, we can get the information that summarization methods based on machine learning are performance better than others. The result of $Centroid$ are better than $Cluster$, mainly because this method use some surface feature. The $Cluster$ method is the worst in our experiments, since this method cluster the same meaning sentences and choose one from cluster, which makes the result ignore the novelty. The $Allan$ method's performance better than $Centroid$ cause this method consider the novelty and importance. Our method $TMTS$ performance better than $Wong$ and $L2RTS$ on Top-1 summary mainly because we use the SMOTE method to train the model, which makes the minorty class can be classified better. But when testing in Top-3, the $L2RTS$ wins the match, because the SVMRanking can gain a better classification model.
From this experiment, we can get the information that summarization methods based on machine learning are performance better than others. The result of $Centroid$ are better than $Cluster$, mainly because this method use some surface feature. The $Cluster$ method is the worst in our experiments, since this method cluster the same meaning sentences and choose one from cluster, which makes the result ignore the novelty. The $Allan$ method's performance better than $Centroid$ cause this method consider the novelty and importance. Our method $TMTS$ performance better than $Wong$ and $L2RTS$ on Top-1 summary mainly because we use the SMOTE method to train the model, which makes the minorty class can be classified better. But when testing in Top-3, the $L2RTS$ wins the match, because the SVMRanking can gain a better classification model.

%
\section{Conclusion and Future}
%

In this paper, we present a novel approach of involving aging theory and SMOTEBoost to timeline summarization. In our approach, we firstly construct features of each sentence, which contains surface feature, importance feature, aging feature, novelty feature and topic feature. Then we treate the multi-document summarization task as pair-wise classification task and generate the training data. At last SMOTEBoost is used to train the model. Experiment results show that our approach performs better compared with other widely used methods.

In the future, we will identify semantic units using other methods since $LSA$ can process synonym but is unable to handle polysemy. And we will also extend our approach to short text such as microblogs and comments.

%Acknowledge%
%\section{Acknowledgements}
%something and somebody should be thanks...



\bibliographystyle{ieeetr}
\bibliography{../../bib/text_summary}



\end{document}