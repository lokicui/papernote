% This is LLNCS.DOC the documentation file of
% the LaTeX2e class from Springer-Verlag
% for Lecture Notes in Computer Science, version 2.4
\documentclass{llncs}
\usepackage{llncsdoc}
%
\begin{document}
\markboth{\LaTeXe{} Class for Lecture Notes in ComputerScience}
{\LaTeXe{} Class for Lecture Notes in Computer Science}
\thispagestyle{empty}
\vspace{2pt}
\vfill

%
\title {Leveraging  Aging theory in topic-focused multi-document timeline summarization} 
\maketitle
%
\section{Abstract}

Topic-focused multi-document summarization plays an important role in helping readers to get the main information from any topic. Many approaches are proposed to generate the timeline summarization, but seldom consider the life circle of each topic. In this paper, we leveraging aging theory to present the sentence feature, and train the classification model with the SMOTEBoost technology. We evaluate our approach on two corpus, one of which is a public data set, the other one is our manual annotation data set. Experiment results show that our method can improve the timeline summarization significantly. 


%
\section{Introduction}
%


Everyday thousands of news stories reporting different events are published on the Internet. These reports are disordered and people have to read most of them to know what is happening which is a time-consuming job undoubtedly. How can we get useful information about an event efficiently? Automatic summarization has been such a method solving this kind of information overloading since Luhn \cite{1958-Luhn-p159-165} proposed it in 1958. And numerous pages have been published in this field, ranging from single document to multiple documents, from extraction to abstraction, from traditional document to web document, email, blog and other types of genre. However, these research work focus on the central idea of document or document set ignoring the temporal characteristics of events. As a result, people cannot catch the changes of events over time efficiently. 

Recent years, topic detection and tracking (TDT) which detects new events from the large scale news stream and tracks them as events going on draws researchers' attention. But it did not display events properly, and people still have to read all the relevant reports to get what they want to know about the event. However, we are still enlightened by its usage of tracking which make us decide to generate a timeline summary consisting of a series of individual small summaries with sentences both important and diverse to help people understand the progress of an event more quickly.

Every event goes through a life cycle of birth, growth, maturity and death, which means that special terms utilized for descripting different events experience a similar life cycle. Aging theory \cite{2003-Chen-p47-59} is a model exploited in event detection task which tracks life cycles of events using energy function. The energy of an event increases when the event becomes popular, and it diminishes with time. In our opinion, it can also be used for summarization to help us find out the daily hot terms of events. Then people can obtain what new changes happen as events going on.

One challenge lie in compute the importance of sentences, which is decided by terms occurring at the documents in keywords-based summarization. But different authors use different words to express the same meaning and lots of synonyms. In order to find the core word in the news without the influence caused by the synonym and polysemy, we use latent semantic analysis (LSA) \cite{1990-Deerwester-p391-407} to handle the dataset. LSA is a robust unsupervised technique for deriving an implicit representation of text semantics based on observed co-occurrence of words to find semantic units of news.

The other challenge is how to handle the sentences. In each topic, only a few sentences will be labled as the summarization sentence, that is, the data set is imbalanced. This situation caused a problem that the training model will prefer the normal sentence. SMOTEBoost \cite{chawla2013smoteboost} combine SMOTE \cite{chawla2011smote} and boost technology in order to impove the precsice for minority class through resamping. This has been proved effictive for imbalanced data set. 

In this paper, we generate news event timeline summary by considering both temporal and semantic characteristics. We first extract the features from five aspects to represent each sentence. Then, classification model is build with SMOTEBoost. Last, we choose sentences from candidates to from the summary and display them with timeline, so that people can track the progress of event easily and quickly.

The remainder of this paper is organized as follows: Section 2 reviews some related works on summarization. We discuss our approach about how to leverage aging theory to gain sentence feature and train the classification model with SMOTEBoost technology in section 3. Our experiments and some discusses are described in section 4. Section 5 presents our conclusions and some future plans.


%
\section{Related Work}
%
\textbf{Topic-focused multi-document summarization(TMS)} aims at gain main information from multiple text about the same topic. There are two ways to achive this goal, one is extract important sentences, the other one is build new sentence to express the key idea. In this paper, we focuse on the former method.

One of the most popular extractive multi-document summarization method is MEAD \cite{2004-Radev-p919-938}, which take term frequency, sentence position, first-sentence overlap to present the feature of each sentece. Xiaojun Wan \cite{wan2007manifold} proposed a extractive approach based on manifold-ranking about the information richness and novelty.

\textbf{Timeline summarization(TS)} gain enough attraction with the development of Topic Track and Detection(TDT). Lots of timeline summarization methods have been developed recently. ETS \cite{2011-Yan-p745-754} formulate the task as an optimization  problem via iterative substitution from a set of sentences with four requirements. Giang Binh Tran\cite{tran2013leveraging} investigate five different sentence feature and leverage SVMRank to optimize the summarization task. Wayne Xin Zhao \cite{zhao2013timeline} take social attention involved to compute the importance. ETTS \cite{yan2011timeline} utilize trans-temporal characteristics to gain the summary. 
 
\textbf{Aging theory} proved to be effictive to track what stage of life cycle for news. Chen et al. \cite{2007-Chen-p1016-1025} \cite{2003-Chen-p47-59} applied this to model the news event's life cycle and utilized the concept of energy to track it. Because our aim is to gain the summary of multi-documents of news domain, we consider aging theory is worth using to extract the feature of sentence.




%
\section{Our Approach}
%

\subsection{Key Concepts}
\textbf{Topic-focused}: What we value most is an event grouped from several web news articles, such as the  "the missing of the malaysia airlines plane" from BBC. These articles show us the cause, the progress and the results about the topic. Most of our summarrization technology's application scenario is working for TDT system. \textbf{Timeline Summaries:} Generally speaking, timeline is one of display form for the summaries. But timeline summaries should show us the progress of this topic not just display according to the time sequence. Under this condition of the requirement, timeline summary of each day should describe the most important thing happened in that day.

We give the formal definition of topic-focused mult-document timeline summarization(TMTS) as follows:

\textbf{Input}: Given a set of documents $D=\{d_1, d_2, \dots, d_n\}$ which should cover progress of the topic in the time span $T=\{t_1, t_2, \dots, t_m\}$. We segment each document to sentences and group them by the date to form sentences $S=\{s_1, s_2, \dots, s_m\}$. 

\textbf{Output}: The TMTS should ouput the summaries along the date and each summary is the main idea of what occured in that day, i.e. $O=\{o_1, o_2, \dots, o_m\}$, where $o_i$ means the summary of sentences from all the documents of that day $s_i$. 

In order to respresent the most important thing happened in that day, the summary should consider the importance, the noviety, the  

\subsection{Sentence Feature Selection}

In order to represent sentence, we extract five kinds of features as flows:

\textbf{Surface feature}: this contains features computed by basic statistics, such as the length of sentence, the counts of nourm words and stop words, the position in this document and parapraf, and wheather it contains person name or not. %$Weight_{surface} = \{Length_{sentence}, Count_{nourms}, Count_{stopwords}, Position_{indocument}, Position_{inparagrph}, Boolean_{containspersonname}\}$

\textbf{Importance feature}: this feature aim to respesent the impantance of this sentence. The weight of sentence is computed through linear combination of term weights with latem senmatic analysis. $Weight_{importance} = TF_i * LSA_i$

\textbf{Aging feature}: We use this feature to show the life cycle of this sentence. The frequency of a word will change as event going on, so we use the association between word  and time interval  to indicate its energy which is defined as follows:
\begin{equation}
  E_{w,t} =F(F^{-1}E_{w,t-1}+\alpha\cdot\chi^2_{w,t})
\end{equation}
where $E_{w,t}$ is the energy of word $w$  in time interval $t$ , and $E_{w,t-1}$  is the energy of word $w$ in time interval $t-1$ , $\alpha$ is the transfer factor, and $\chi^2_{w,t}$  is the contribution degree of word  at the time interval $t$, which can be computed as presented in \cite{2000-Swan-p49-56}. 

However, no words descripting a special event point will retain popular forever, they will decay over time. In order to represent the word's life span realistically, we cut down the energy of word by a decay factor β at the end of every time interval. And if the decayed energy value became negative, we change it to 0.

According to the description above, if the energies of some words increase greatly, we can draw a conclusion that there is a hot event spot. So we need to calculate the variance of word energy next. Here we use standard deviation:
\begin{equation}
Var_{w,t} = \sqrt{ \frac{1}{N} \sum_{t \in period}(E_{w,t}- \overline{E_{w}})^2}
\end{equation}
where $N$ is the number of time intervals during the given period, $E_{w,t}$  is the energy of word $w$  in time interval $t$ , $\overline{E_{w}}$  is the average energy during the period, and $Var_{w,t}$  is the variance of $w$ .
Then each word will be assigned a new weight besides the traditional TFIDF which can be defined as:
 \begin{equation}
NewWeight(w)=TF*IDF_{w} + \mu \cdot Var_{w}
\end{equation}
This kind of new weight can help us identify both central and hot information, so people can capture the main line and new changes of events simultaneously.

\textbf{Noviety feature}: 

\textbf{Topic feature}: 


\subsection{Model Trainning}

With the help of labled data, we convert this summarization task to pairwise classification problem. The positive data is sentences labeld to summary, otherwise is negative. 

Because the count of summary sentence is much less than the normal sentence, the train data set is unblanced. In order to reduce this reflect, SMOTEBoost method is used to train the classification model.

%
\section{Evaluation}
%

\subsection{Evaluation metric}

Here we use ROUGE toolkit \cite{2004-Lin-p74-81} , which is officially applied by Document Understanding Conference (DUC) for document summarization performance evaluation, to evaluate the experimental results and compare these algorithms with each other. The summarization quality is measured by counting the number of overlapping units, such as N-gram, word sequences, and word pairs between the auto-generated summary and the manual summary. Several automatic evaluation methods are implemented in ROUGE, such as ROUGE-N, ROUGE-L and ROUGE-W, each of which can generate three scores (recall (R), precision (P) and F-measure). Take ROUGE-N as an example:

\begin{equation}
R=\frac{\sum_{s \in manual } \sum_{N-gram \in s} Count_{match}(N-gram) } { \sum_{s \in manual} \sum_{N-gram \in s} Count(N-gram)  } \end{equation}
\begin{equation}
P=\frac{\sum_{s \in auto } \sum_{N-gram \in s} Count_{match}(N-gram) } { \sum_{s \in auto} \sum_{N-gram \in s} Count(N-gram)  }
\end{equation}
\begin{equation}
F-measure = \frac{2PR}{P+R}
\end{equation}

Where $N$ stands for the length of the $N-gram$, $Countmatch(N-gram)$ is the maximum number of N-grams co-occurring in the auto-generated summary and the manual summary. Here we execute our evaluation in terms of ROUGE-1. And since our summary consists of a series of individual small summaries, we use the average ROUGE score as the final score.

\subsection{Methods to compare}

We start our experiment with some preprocessing like indexing, filtering out the stop words and segmenting news documents into sentences. Then we perform our method to the data set and generate a timeline for each event we choosing. We also implement some widely used multi-document summarization methods as the baselines.

$Centroid$ extracts sentences based on centroid value, positional value and first-sentence overlap.

$Cluster$ considers that there are different themes in an event, so it first clusters similar sentences together into different clusters and then selects one representative sentence from each main cluster.

$Allan$ is a similar timeline system from different aspect proposed by Allan et al., dividing sentences into on-event and off-event while ranking them with useful and novelty.

$ETS$ is 

$L2RTS$

\subsection{Experiment on public data set}

The public data set form Giang Binh Tran\cite{tran2013leveraging} is used in this research. This data set collects 17 timelines in 9 topics and it obtained 4650 news articles. The data set provider take the news editor's event summary as the topic summary, and some of labled summary is created by themselves. So there are many summary sentences are not extracted from the document, which brings extra problem about evaluate the results from TMTS.

Pictures and evaluation will available until Wednesday.


\subsection{Experiment on muannl labeld data}

We also create a data set, which come from the TAC2010. This data set contains 906 documents around 46 topics from the New York Times, the Associated Press, the Xinhua News Agency newswires, and the Washington Post News Service, however, it is not for summarization originly because of no summary label. In order to gain the relative accurate lable result, we distribute these documents to five people to label, then the highest count of sentence is the summary.

Pictures and evaluation will available until Wednesday.



%
\section{Conclusion and Future}
%

In this paper, we present a novel approach for evolove aging theory and SMOTEBoost to timeline summarization. In our approach, we firstly construct feature of each sentence, which contains surface feature, importance feature, aging feature, noviety feature and topic feature. Then we treate the multi-document summarization task as pair-wise classification task and generate the training data. At last SMOTEBoost is used to train the model. Experiment results show that our approach performs better compared with other widely used methods.

In the future, we will identify semantic units using other methods for $LSA$ can process synonym but is unable to handle polysemy. And we will also extend our approach to short text such as microblogs and comments.

%Acknowledge%
\section{Acknowledgements}
something and somebody should be thanks...



\bibliographystyle{ieeetr}
\bibliography{../../bib/text_summary}


\end{document}