\documentclass{article}
\usepackage{CJKutf8}
\begin{document}
\begin{CJK}{UTF8}{gkai}


\title{新闻的特征抽取方法}
\maketitle

\section{整篇新闻的特征}

\subsection{TF * IDF}

将文章里的词都提取出来,计算每个新闻的tf*idf 来得到一个向量


\section{句子的特征}
\subsection{Centroid-based summarization of multiple documents}
\cite{2004-Radev-p919-938}

Centroid value

The centroid value Ci for sentence Si is computed as the sum of the centroid values Cw;i of all words in the sentence. For example, the sentence ‘‘President Clinton met with Vernon Jordan in January’’ would get a score of 243.34 which is the sum of the individual centroid values of the words (clinton = 36.39; vernon = 47.54; jordan = 75.81; january = 83.60).
\begin{equation}
C_i = \sum_wC_{w,i}
\end{equation}

Positional value

The positional value is computed as follows: the first sentence in a document gets the same score Cmax as the highest-ranking sentence in the document according to the centroid value. The score for all sentences within a document is computed according to the following formula:
\begin{equation}
P_i = \frac{n-i+1}{n}*C_{max}
\end{equation}

First-sentence overlap

The overlap value is computed as the inner product of the sentence vectors for the current
sentence i and the first sentence of the document. The sentence vectors are the n-dimensional
representations of the words in each sentence, whereby the value at position i of a sentence vector
indicates the number of occurrences of that word in the sentence.
\begin{equation}
F_i = \overrightarrow{S_1}\overrightarrow{S_i}
\end{equation}

\end{CJK}

\bibliographystyle{ieeetr}
\bibliography{../../bib/text_summary}
\end{document}
