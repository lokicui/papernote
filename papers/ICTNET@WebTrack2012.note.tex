\documentclass{article}
\usepackage{CJKutf8}
\begin{document}
\begin{CJK}{UTF8}{gkai}
%Test it 中文如何?

%ICTNET
ICTNET在2012 web track的试验方法

1.建索引
用新的网页解析器,将网页中的广告和spam去掉,抽取了 trec-id,title, url,正文,锚文本
2.锚文本的使用
对锚文本的检索有效的提升了搜索效果(参见ICTNET2011);锚文本的数据采用Lemor官方提供的锚文本数据集。采用了map-reduce的方法来统计各url中的锚文本
3. 建立索引
1)使用天矶的索引器,对trec-id,url,title,正文,锚文本,spam value进行索引。2)在10台机器上使用天机2进行索引,10个小时就搞定了。
4.排序模型--BM25
使用boolean模型和bm25模型的结合。
在短的域上(title,锚文本)使用OR形式,对长的域用AND形式。最后对所有的备选文档计算bm25的值,然后加和排序。
5.排序模型--learning to rank
采用来RANKBOOST的模型,训练数据是web trec 2009和2010的结果集,经过5-folds的交叉验证后直接用于排序。
6.结合wikipedia结果
用wiki中的结果作为top1,用其他的结果作为补充。
7.结果分析
一共三轮结果。
a,采用原来的文本抽取器,BM25,wiki数据,作为baseline
b, 采用新文本抽取器,BM25,WIKI,在ERR@20上比a好,其他都不如a,说明正文提取对于准确率有帮助,但是降低了召回率。
c,采用排序学习方法,效果比a差很多,可能是由于训练样本太少造成。
8.综上
简单的排序模型即可,关键在于如何筛选掉错误的备选。
\end{CJK}
\end{document}
